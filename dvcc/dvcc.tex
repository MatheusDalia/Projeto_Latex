\documentclass[10pt,a4paper]{article}

\usepackage[brazil]{babel}
\usepackage[utf8]{inputenc}
\usepackage[T1]{fontenc}

\usepackage[a4paper,top=2cm,bottom=2cm,left=3cm,right=3cm,marginparwidth=1.75cm]{geometry}


\usepackage[colorinlistoftodos]{todonotes}
\usepackage[colorlinks=true, allcolors=blue]{hyperref}

\title{IF674 - Infra-Estrutura de Hardware}
\author{Daniel Victor Cintra Cavalcante}
\date{May 2019}

\begin{document}

\maketitle

\section{Introdução}
O Curso de Infra-Estrutura de Hardware  visa dar uma visão geral dos componentes de um computador, quais sejam: processador, sistema de memória (memória principal e memória cache), Entrada e Saída e Barramentos.  Nesta disciplina os princípios de funcionamentos de cada um dos componentes acima serão apresentados e o aluno terá possibilidade de sedimentar estes conceitos seja pelo projeto de uma versão simples do componente, ou seja pela simulação do mesmo através de ferramenta de simulação.

\section{Relevância}
A disciplina trabalha conceitos básico, o que caracteriza uma CPU, e o aluno terá a oportunidade de projetar uma CPU bem simples de forma a compreender melhor o seu funcionamento. Além dos conceitos básicos, serão apresentados conceitos avançados como pipeline e super-escalares, técnicas usadas nos processadores comerciais atuais e que garantem um grande aumento no desempenho da máquina. Os computadores estão cada dia mais presentes no nosso dia, como isso, a importância das disciplinas como infraestrutura de hardware e outras relacionadas veem aumentando cada dia mais, tornando-se importantes peças para o desenvolvimento tecnológico das maquinas.
\begin{figure}[h!]
    \centering
    \includegraphics[scale=0.3]{MOS_Technologies_large}
    \caption{6510 CPU (long chip, lower left) and the 6581 SID (right) - Licença - \cite{img}}
    \end{figure}
 \newline
De uma forma geral, o objetivo é fazer com que o aluno passe a entender os diversos aspectos de projetos e implementação de computadores e use este conhecimento de forma a auxiliar em tarefas de sua vida profissional abrangendo desde a definição de computadores a comprar para uma determinada tarefa, até projetos de máquinas. 

\newpage

\section{Relação com outras disciplinas}
Infraestrutura de Hardware faz parte da tríade software, hardware e comunicação, que é a base da construção de praticamente qualquer sistema de computação atual.

\begin{table}[h]

\centering

\label{my-label}

\begin{tabular}{|p{5cm}|p{7cm}|}

\hline

I-Estrutura de Software - IF677 & \begin{tabular}[c]{@{}p{7cm}@{}} O objetivo aqui é apresentar os conceitos e sistemas de software básicos de um computador, que compreende a introdução aos sistemas concorrentes e aos sistemas operacionais, sejam eles mono-computador ou distribuídos. \end{tabular}\\ \hline
I-Estrutura de Hardware - IF674  & \begin{tabular}[c]{@{}p{7cm}@{}} O que se tem vista aqui, é a parte de componentes do computador, e o funcionamento de cada elemento. \end{tabular}\\ \hline
I-Estrutura de Comunicação. - IF678  & \begin{tabular}[c]{@{}p{7cm}@{}} Trabalha o entendimento dos diversos aspectos de projetos e implementação de redes de computadores, além de como funcionam a internet e os diversos protocolos de comunicação existentes.  \end{tabular}\\ \hline
\end{tabular}
\end{table}
Além de fazer parte desta tríade, a disciplina está fortemente ligada com Sistemas Digitais (IF675), que tem como objetivo dar ao aluno conhecimentos de circuitos lógicos digitais combinacionais e seqüenciais cobrindo desde dispositivos digitais de pequena complexidade SSI, até a implementação de circuitos de média complexidade MSI.

\bibliographystyle{plain}
\bibliography{references}
\nocite{CO}
\nocite{OP}
\nocite{AO}
\end{document}
