\documentclass{article}
\usepackage[utf8]{inputenc}

\title{IF687 - Introdução a Multimídia}
\author{Bruno de Melo Costa }
\usepackage[brazil]{babel}
\date{Maio 2019}

\usepackage{natbib}
\usepackage{graphicx}

\begin{document}

\maketitle

\section{Introdução}
Introdução a Multimídia é uma disciplina que estuda, cria e programa percepções sensoriais, como visão e audição, de modo a criar uma nova perspectiva a determinados dados, para torná-los mais amigáveis e aperfeiçoar  o entendimento.\citep{referenciasdisciplina} Alguns de seus tópicos são:\newline\newline
\textbf{Realidade Virtual:} é relacionada com o estudo de criação de uma perspectiva diferente da perspectiva do mundo real.\citep{referenciasdisciplina1}\newline
\textbf{Realidade Aumentada:} é relacionada com o estudo de como alterar a percepção do mundo real, seja adicionando objetos virtuais a ele, seja alterando sua forma.\citep{referenciasdisciplina2}

\begin{figure}[ht]
    \centering
    \includegraphics[scale=0.15]{imagem.jpg}
    \caption{Exemplo de mudança de percepção}\cite{figura}
    \label{fig:my_label}
\end{figure}

\section{Relevância}
A relevância da Introdução a Multimídia é ampla. Essa disciplina interage com as percepções do homem na realidade inserida. Conectando essa realidade com algum dispositivo, é possível criar um modo de manuseá-lo com uma precisão formidável. Além disso, adicionando objetos virtuais no mundo real pode-se auxiliar a percepção de diversos corpos dificilmente observáveis ao olho humano. Desse modo, a área de estudo da Introdução a Multimídia modifica a perspectiva do homem para interagir direta e facilmente com o meio virtual com uma maior precisão e agilidade.


\section{Relação Com Outras Disciplinas}
\begin{tabular}{c|c}
  Disciplina   &  Relações \\
   & \\
  MA531 - Álgebra Vetorial Linear & Fornece os recursos essenciais para a \\para Computação & manipulação dos dados visuais. \\ \\
  IF755 - Realidade Virtual & Ensino aprofundado da Introdução a Multimídia. \\ \\
  IF669 - Introdução a Programação & Ensina como manipular dados virtuais.	
\end{tabular}

\bibliographystyle{plain}
\bibliography{bmc4}
\end{document}