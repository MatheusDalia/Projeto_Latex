\documentclass{article}
\usepackage[utf8]{inputenc}

\title{IF754 - Computação Musical}
\author{Matheus Dalia}
\date{April 2019}

\usepackage{natbib}
\usepackage{graphicx}

\begin{document}

\maketitle

\section{Introdução}
"O som pode ser considerado como uma pressão variável do ar que desloca no tempo. Suas características subjetivas, como "soa", dependem da maneira como a pressão varia." \cite{techMIT}

A disciplina Computação Musical visa a formação de profissionais capazes de escrever programas de ação multimídia capazes de se adequar aos meios computacionais disponíveis atualmente. Durante o curso, será exigido um conjunto mínimo de domínios de conhecimento dentro da ciência da computação. Esta disciplina oferece aos alunos a possibilidade de complementar seus conhecimentos relativos à natureza da forma sonora, aos algoritmos para a síntese e processamento de sons digitais, e às técnicas de representação e manipulação de informações  musicais, incluindo wave, MIDI, MP3, RealAudio, etc.

Para poder cursar a referida disciplina, o aluno deverá dominar, ou estar em vias de dominar, alguns fundamentos básicos de computação ensinados nas seguintes disciplinas: Algoritmos e estruturas de dados (pré-requisito) e Linguagens de Programação 3 (co-requisito). Ademais, a ementa da cadeira deixa claro de que não é preciso ter conhecimento musical prévio para cursar a cadeira.

\begin{figure}[h!]
\centering
\includegraphics[scale=0.3]{musescore.png}
\caption{O software musical MuseScore}
\label{fig:Peixe}
\end{figure}

\section{Relevância}
"Eu não acho que alguém não gosta de música."\cite{science}

A disciplina Computação Musical consta como uma cadeira eletiva no curso de Ciências da Computação da UFPE. Com o propósito de auxiliar no processo de educação e produção musical, esta cadeira é um exemplo perfeito da versatilidade do curso de Computação.

Um ponto positivo dessa disciplina é fomentar a capacidade de "representar e manipular o som como dados de computador, o que abre as portas para que um compositor / pesquisador trabalhe com som em um nível mais íntimo."\cite{represent}

\section{Relação com outras disciplinas}

\begin{table}[h]
 \centering
% distancia entre a linha e o texto
 {\renewcommand\arraystretch{1.25}
 \caption{ }
 \begin{tabular}{ l l }
  \cline{1-1}\cline{2-2}  
    \multicolumn{2}{|p{4.050cm}|}{Disciplina \centering }
  \\  
  \cline{1-1}\cline{2-2}  
    \multicolumn{1}{|p{4.050cm}|}{IF672 - Algoritmos e Estruturas de Dados} &
    \multicolumn{1}{p{4.050cm}|}{Essa disciplina é um pré-requisito para poder fazer Computação Musical. Em Algoritmos, estuda-se estruturas de dados para que se possa aprender a escrever programas mais eficientes. Portanto, percebe-se que é uma cadeira fundamental para o desenvolvimento de softwares, prática que é o cerne do conteúdo aplicado de Computação Musical}
  \\  
  \hline

 \end{tabular} }
\end{table}





\bibliographystyle{plain}
\bibliography{mdaa}
\end{document}
