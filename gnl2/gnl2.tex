\documentclass{article}
\usepackage[utf8]{inputenc}

\title{IF738-Redes de Computadores}
\author{Gabriel Nogueira Leite }
\date{\vspace{-5ex}}
\date{Abril 2019}

\usepackage{natbib}
\usepackage{graphicx}
\usepackage[brazil]{babel}

\begin{document}

\maketitle

\section{Introdução}
A disciplina tem como foco principal fornecer ao aluno uma visão abrangente do ensino e estudos da área de  redes de computadores, que se trata de uma estrutura de computadores e dispositivos conectados através de um sistema de comunicação com o objetivo de compartilharem informações e recursos entre si. O Centro de Informática da UFPE oferece essa disciplina, os assuntos discutidos e estudados pela disciplina são, escolhidos pelos docentes em conjunto aos interesses dos alunos.
\cite {referenciaDisciplina1, referenciaDisciplina2}
Apesar da opção de escolha dos tópicos, existem alguns essenciais e obrigatórios, dentre eles estão:

\begin{itemize}
   \item Modelo de Referência OSI: O modelo de interconexão de sistemas abertos (modelo OSI) é um modelo conceitual que caracteriza e padroniza as funções de comunicação de um sistema de telecomunicações ou de computação sem considerar sua estrutura e tecnologia interna subjacente.
   \cite{referenciaDisciplina3}
   Seu objetivo é a interoperabilidade de diversos sistemas de comunicação com protocolos padrão. O modelo divide um sistema de comunicação em camadas de abstração;
    \cite {networkWorld}
   \item Tipos de redes: Existem vários tipos diferentes de redes de computadores. Podem ser caracterizadas por seu tamanho e por sua finalidade, o tamanho de uma rede pode ser expresso pela área geográfica que ocupam e pelo número de computadores que fazem parte da redes podendo abranger desde um pequeno conjunto de dispositivos até redes globais.
   \cite{networks}
 \end{itemize}

\begin{figure}[ht]
\centering
\includegraphics[scale=0.17]{redes.jpg}
\caption{Redes de Computadores}
\cite{imagem}
\label{fig: redes.jpg}
\end{figure}

\section{Relevância}
A informação é o alicerce para uma comunicação eficaz. A comunicação é a moeda universal que nos une e impulsiona nossas operações diárias. Rede de computadores é um favorito entre muitas empresas. Essa tecnologia tradicional provou ser um meio eficaz para melhorar a comunicação proficiente, flexível e simplificada, maximizando a produtividade e os recursos.
\cite{networkImportance}
Por conta desses fatores a disciplina de redes de computadores é fundamental para a introdução do discente a essa área que demanda muitos conhecimentos por ser uma vasta área de conhecimento.

\section{Relações interdisciplinares}
\begin{tabular}{|c|l|}
\hline
Disciplinas relacionadas  & \multicolumn{1}{c|}{Relações} \\ \hline
\begin{tabular}[c]{@{}c@{}}IF685 \\  Gerenciamento de \\ Dados e Informação\end{tabular} & \begin{tabular}[c]{@{}l@{}}Tem como objeto de estudo o gerenciamento de diferentes \\tipos de redes incluindo a de computadores. Todavia, pos-\\sui da mais ênfase nos protocolos, no que tange a área de\\ redes de computadores, por exemplo TCP e IP.
\cite{gerenciamentoDados}
\end{tabular}                       \\ \hline
\begin{tabular}[c]{@{}c@{}}IF678\\  Infra-estrutura de \\ Comunicação\end{tabular} & \begin{tabular}[c]{@{}l@{}}As duas disciplinas estudam métodos para comunicação\\ via computação. Portanto possuem temas de estudo em\\ comum, como: Modelo OSI e introdução à redes de com-\\putadores.
\cite{infraComunicacao}
\end{tabular} \\ \hline
\end{tabular}

\bibliographystyle{plain}
\bibliography{references}
\end{document}